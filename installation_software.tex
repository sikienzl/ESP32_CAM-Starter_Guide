\section{Installation of the required Software}
This section describes the required steps, how the ESP-IDF can be installed under Linux.


\subsection{Prerequisites}

First of all you have to install the required dependencies on your linux distribution.

\begin{itemize}

\item Ubuntu and Debian:
\begin{lstlisting}[language=bash]
sudo apt-get install git wget flex bison gperf python3 python3-pip python3-setuptools cmake ninja-build ccache libffi-dev libssl-dev dfu-util libusb-1.0-0
\end{lstlisting}

\item CentOS 7 \& 8:
\begin{lstlisting}[language=bash]
sudo yum -y update && sudo yum install git wget flex bison gperf python3 python3-pip python3-setuptools cmake ninja-build ccache dfu-util libusbx
\end{lstlisting}

\item Arch:
\begin{lstlisting}[language=bash]
sudo pacman -S --needed gcc git make flex bison gperf python-pip cmake ninja ccache dfu-util libusb
\end{lstlisting}

\end{itemize}

\subsection{Get ESP-IDF}
To download the ESP-IDF, run the following command in a terminal:
\begin{lstlisting}[language=bash]
mkdir -p ~/esp
cd ~/esp
git clone --recursive https://github.com/espressif/esp-idf.git
\end{lstlisting}

\subsubsection{Install the ESP-IDF tools}
Now you can run the following command in a terminal to install the tools of ESP:
\begin{lstlisting}[language=bash]
cd ~/esp/esp-idf
./install.sh esp32
\end{lstlisting}

\subsection{Set up of the environment variables}
The setup does not automatically set up the PATH environment variable. This section describes
how to set the environment variables temporary or as an alias. 

\subsubsection{Set up environment variables temporary}
Below you can enter in a terminal the following line to setup the environment variables of the esp idf temporary:
\begin{lstlisting}[language=bash]
. $HOME/esp/esp-idf/export.sh
\end{lstlisting}

\fcolorbox{red}{pink}{{\fontencoding{U}\fontfamily{futs}\selectfont\char 66\relax} Don't forget the space between the leading dot and the path!}

\subsubsection{Set up environment variables as an alias}
If you want to create an alias for your shell profile (\texttt{.profile}, \texttt{.bashrc}, \texttt{.zprofile}, etc), you can add the following line:
\begin{lstlisting}[language=bash]
alias get_idf='. $HOME/esp/esp-idf/export.sh'
\end{lstlisting}

Now you have to restart the terminal session or you can execute \texttt{source  [path to profile]} to refresh the configuration. 

